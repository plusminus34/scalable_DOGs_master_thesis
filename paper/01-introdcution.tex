\section{Introduction}
There are myriad of ways to deform a planar sheet without stretching or tearing it. One can bend it, fold it, or combine the two. Though both represent classes of isometries on planar surfaces, they are different by nature, and historically speaking there is some dichotomy in the study of the two; Bending, or smooth isometries, are typically studied in differential geometry [cite], whereas folds along straight lines are often explored in the field of computational origami [cite]. Curved folded surfaces [cite the demaine review] combines these two world by folding a surface along a curve, which in turn necessitates global bending around the crease.

%They date to 1920's but we only know what happen along one fold, or analyze specific set of examples, and don't know how typical models even fold. Computer simulations of that can help, but are somehow legging behind what is mostly experimented by hand, in a process that is in fact time consuming, often involving using a pen bla and da da da.
