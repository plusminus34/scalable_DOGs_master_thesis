\section{Introduction}
There are myriad of ways to deform a planar sheet without stretching or tearing it. One can bend it, fold it, or combine the two. Folding and bending are isometries that are different by nature, and historically speaking there is some dichotomy in the study of the two; Smooth isometries are typically studied in differential geometry \cite{do_carmo}, whereas folds along straight lines are often explored in the field of computational origami \cite{origami_book}. Curved folded surfaces \cite{huffman} results of a combination of the two, as folding an inextensible sheet along a curve necessitates global bending around the crease. \\
Building curved folded sculptures is a manual time and consuming process, often done using an empirical trial and error approach  as little theory is known \cite{curved_review,huffmann_reconstructing}. Contrary to classical origami, folding instructions are typically not produced and multiple  creases needs to fold and bend simultaneously \cite{StringActuated:2017}, further complicating the process. In practice, artists often pre-crease the paper using a ball burnisher or a CNC plotter before carefully bending it.



% The beautiful art of curved folded sculptures is almost a hundred years old, dating back to the 1920's works of Josef Albers in the Bauhaus art school \cite{josef_albers_thesis}, and continuing with the investigations of David Huffman and Ron Resch in the 1970's \cite{huffman,resch1974portfolio}. 

% Many of these early works, are mathematically not yet fully understood. We know what happens in one fold, but multiple ones still forms a challenge and we only have specific investigation for some cases. Creating them is , using paper and They date to 1920's but we only know what happen along one fold, or analyze specific set of examples, and don't know how typical models even fold. Computer simulations of that can help, but are somehow legging behind what is mostly experimented by hand, in a process that is in fact time consuming, often involving using a pen bla and da da da.
