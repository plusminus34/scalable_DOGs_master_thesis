\section{Conclusions and future work}
This paper is a first step towards unhindered freeform modeling of curved folded surfaces. Basing our models on DOGs \shortcite{rabi18} allows us to capture the full set of curved folded deformations, and our discretization in \secref{sec:folding} together with the folding algorithm in \secref{sec:implementation} allows us to steer the modeled deformations towards those that simultaneously fold and bend crease curves. Our deformation algorithm is able to model bending and folding of complicated crease patterns by merely using positional constraints, making it highly suited for exploration of new curved folded surfaces. We supply further optional objectives to constrain dihedral angles and mountain/valley assignment in \secref{sec:folding_angles_mountain_valley}, giving designers additional expressiveness. \\
Similar to other works on modeling DOGs \shortcite{rabi18,rabi2018shape}, the most obvious limitation of our algorithm is speed. Our optimization framework allows us to interactively model up to $2000$ vertices. We leave scaling of the optimization to future work, possibly by using a multigrid solver on the DOG grids. Throughout the design, we found that we lacked tools and objectives to enforce symmetry. In particular, we would like to look at folding of curved symmetric plane wallpappers and tessellations \cite{demaine_lens,mundilova2019mathematical}. Finally, we note that we model deformations of a given fixed input crease pattern. Optimizing and changing an input crease pattern, as done in origami modeling tools \cite{tachi2010freeform}, could offer new and exciting ways to discover and design curved folded surfaces. 