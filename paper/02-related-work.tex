\section{Related work}
\subsection{Modeling developable surfaces}
A smooth surface is called a \textit{developable surface} if it is locally isometric to the plane, or equivalently has zero Gaussian curvature. Though well understood mathematically \cite{do_carmo,spivak,computational_line}, there is a plethora of work on modeling developable surfaces, which has been proven to be a challenge. The primary difficulty lies in using a discrete model that is able to capture the full set of deformations keeping a surface developable. These include extrinsic deformations as well as intrinsic deformations. The latter stretch the surface while keeping it developable, and is used for geometry exploration tasks where the size and shape of the flattened developable surface is unknown \cite{conical,pottmann_new,rabi2018shape}. A failure of a discrete model to discretize the full range of smooth deformations is often termed \textit{locking} \cite{solomon,locking1}. Ruling based discrete models \cite{conical,curved_folding_kilian,pottmann_new,stein_dev} lock the user to a partial set of extrinsic deformations while isometry based simulations \cite{grin_shells,shells, goldenthal2007efficient,froh_botsch} do not model intrinsic deformations by design, but are also prone to locking of various bending deformations \cite{locking1,locking2}, and is often coupled with dynamic remeshing \cite{narain2012adaptive,StringActuated:2017,Narain,SchreckEG2017}. \MiR{should I explicitly say that remeshing is complicated and is also not backed up by any theory?}
\subsection{Curved folding}
%The beautiful art of curved folded sculptures is almost a hundred years old, dating back to the 1920's works of Josef Albers in the Bauhaus art school \cite{josef_albers_thesis}, and continuing with the investigations of David Huffman and Ron Resch in the 1970's \cite{huffman,resch1974portfolio}. Many of these early works are mathematically not yet fully understood. %TODO. (some work has been done on specific models...$

%  We know what happens in one fold, but multiple ones still forms a challenge and we only have specific investigation for some cases. Creating them is , using paper and They date to 1920's but we only know what happen along one fold, or analyze specific set of examples, and don't know how typical models even fold. Computer simulations of that can help, but are somehow legging behind what is mostly experimented by hand, in a process that is in fact time consuming, often involving using a pen bla and da da da.