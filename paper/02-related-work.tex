% !TEX root =  CurvedFoldedDogs.tex

\section{Related work}
\subsection{Modeling developable surfaces}
A smooth surface is called a \textit{developable surface} if it is locally isometric to the plane, or equivalently has zero Gaussian curvature. Though well understood mathematically \cite{do_carmo,spivak,computational_line}, computer aided modeling of developable surfaces has been proven to be a challenge and is an active research area. The primary difficulty lies in finding a discrete model that is able to capture the full set of deformations while keeping the surface developable. Deformations can be extrinsic as well as intrinsic. The latter stretch the surface while keeping it developable, and are used for geometry exploration tasks where the size and shape of the flattened developable surface is unknown \cite{conical,pottmann_new,rabi2018shape}. A failure of a discrete model to represent the full range of smooth deformations is often termed \emph{locking} \cite{solomon,locking1} and is the bane of most discrete developable models. Ruling based models \cite{conical,curved_folding_kilian,pottmann_new,stein_dev,solomon} are limited to a partial set of extrinsic deformations, while isometry based methods \cite{grin_shells,shells, goldenthal2007efficient,froh_botsch} do not model intrinsic deformations by design, and are also prone to locking of various bending deformations \cite{locking1,locking2}, and often must be coupled with dynamic remeshing \cite{narain2012adaptive,StringActuated:2017,Narain,SchreckEG2017}.

Our work is based on modeling a developable surface as a discrete orthogonal geodesic net (DOG) \cite{rabi18}, a model that has been shown both theoretically and empirically to avoid extrinsic and intrinsic deformation locking. 
We further rely on \cite{rabi2018shape} to explore the shape space of DOGs, but we replace their Laplacian flow based deformation with a sequential quadratic programming (SQP) based algorithm, detailed in \secref{sec:implementation}. 

In addition to the above literature, it is worth mentioning the large body of works focused on designing developable surfaces by fitting a developable to a target shape \cite{pottmann_approx,pottmann_new,stein_dev} or to a set of boundary curves \cite{sheffer,Frey1,Frey2,bo2019multi}. Rather than modeling freeform developable deformations, these works deal with the challenges of ambiguity when fitting to sparse inputs, as well as approximation quality. 

\subsection{Curved folding}
Curved folded sculptures are beautiful works of art almost a hundred years old, dating back to the 1920's works of Josef Albers in the Bauhaus art school \cite{josef_albers_thesis}, and continuing with the investigations of David Huffman and Ron Resch in the 1970's \cite{huffman,resch1974portfolio}. This direction in art, though intimately linked to the mathematics of developable surfaces, is mostly driven by physical experiments with paper \cite{curved_review}. As opposed to smooth developable surfaces \cite{do_carmo} or straight fold origami \cite{origami_book}, the mathematics of curved folding is lagging behind the manual craft and mostly concerns the local behavior of a single folded curved crease \cite{duncan_folded,mathematical_omnibus,curved_review}. Notable crease patterns, such as the Huffmann Tower \cite{huffman2,huffmann_reconstructing}, are not yet understood \cite{demaine2018conic} and the known mathematics on the folding and bending of multiple curved creases is limited to few particular cases, coupled with a specific folding movement and guided by fixed rulings \cite{demaine_lens, demaine2018conic}. In essence, we do not know much about which crease patterns can fold, and we do not know in which ways they can fold. Unlike straight origami, there are often infinitely many ways of folding and bending a curved crease pattern by varying the dihedral angles as well as the developable rulings along the different creases. 

Curved folding was introduced to the geometry processing community by the work of Kilian and colleagues \shortcite{curved_folding_kilian}, where the authors devised an algorithm to reconstruct scanned paper curved folded surfaces by estimating their ruling directions. Several works on modeling curved folded surfaces focus on a given subset of folding deformations, such as planar creases generated by reflection \cite{mitani2012column,Mitani_ref}, surfaces generated by rotational sweeps \cite{mitani2009design} or surface folded with a fixed ruling pattern \cite{pottmann_new}. In \cite{StringActuated:2017} the authors simplify the process of fabrication for a wide range of curved folded models using a network of strings, solving the problem of which surface points to pull in order to actuate a folding movement. To model the folding deformation the authors of \cite{StringActuated:2017} employ the model of \cite{botsch2006primo} with the remeshing algorithm in \cite{narain2012adaptive}. Their deformation is guided by mountain/valley assignments of all creases, which are provided as input, as well as prescribed soft constraints on folding angles, as well as a bending objective. We note that prescribing dihedral angles on a single curve results in an undetermined system, while prescribing the folding angles of multiple creases often results in an overdetermined system. There are infinitely many ways to fold a surface with the same prescribed folding angles \cite{more_on_paper,duncan_folded}, however dihedral angles across multiple folds must be compatible with each other \cite{demaine2018conic}. 

To the best of our knowledge, the first freeform handle based system for curved folding deformations was introduced by \cite{rabi2018shape}.  The Rabinovich et al.~\shortcite{rabi2018shape} employed multiple DOGs together with a set of stitching and flattability constraints to model curved crease deformations. Our work builds and extends upon \cite{rabi2018shape} by deriving a discrete characterization for a fold along a crease, allowing us to devise a point handle based editing system that ensures folding of the creases of a given pattern rather than smoothly ignoring them, all without requiring the user to specify dihedral angles or mountain/valley assignments (see \figref{fig:folded_and_not_folded}). In cases where more explicit control is desired, we also derive simple quadratic constraints to control dihedral angles along folds and mountain/valley assignments, a degree of freedom which is often only available along one curved crease (see \figref{fig:MV_bias_modeling} and \figref{fig:dihedral_editing}).