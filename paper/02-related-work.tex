\section{Related work}
\subsection{Modeling developable surfaces}
A smooth surface is called a \textit{developable surface} if it is locally isometric to the plane, or equivalently has zero Gaussian curvature. Though well understood mathematically \cite{do_carmo,spivak,computational_line}, there is a plethora of work on modeling developable surfaces, which has been proven to be a challenge. The primary difficulty lies in using a discrete model that is able to capture the full set of deformations keeping a surface developable. These include extrinsic deformations as well as intrinsic deformations. The latter stretch a surface while keeping it developable, and is used for geometry exploration tasks where the size and shape of the flattened developable surface is unknown \cite{conical,pottmann_new,rabi2018shape}. A failure of a discrete model to discretize the full range of smooth deformations is often termed \textit{locking} \cite{solomon,locking1} and is the bane of most discrete developable models. Ruling based models \cite{conical,curved_folding_kilian,pottmann_new,stein_dev,solomon} are limiteds to a partial set of extrinsic deformations while isometry based methods \cite{grin_shells,shells, goldenthal2007efficient,froh_botsch} do not model intrinsic deformations by design, but are also prone to locking of various bending deformations \cite{locking1,locking2}, and are often coupled with dynamic remeshing \cite{narain2012adaptive,StringActuated:2017,Narain,SchreckEG2017}. \MiR{should I explicitly say that remeshing is complicated and is also not backed up by any theory?} Our work is based on modeling a developable surface as a discrete orthogonal geodesic net (DOG) \cite{rabi18}, a model that have been shown both theoretically and empirically to avoid extrinsic or intrinsic deformations locking. We further rely on \cite{rabi2018shape} to deform and explore the shape space of DOGs, but replace the authors' Laplacian flow deformation with an SQP algorithm as further detailed in \secref{sec:implementation}. We also mention the large body of works on designing developable surfaces by fitting a developable to a target shape \cite{pottmann_approx,pottmann_new,stein_dev} or to a set of boundary curves \cite{sheffer,Frey1,Frey2,bo2019multi}, instead of modeling freeform developable deformations.

\subsection{Curved folding}
The beautiful art of curved folded sculptures is almost a hundred years old, dating back to the 1920's works of Josef Albers in the Bauhaus art school \cite{josef_albers_thesis}, and continuing with the investigations of David Huffman and Ron Resch in the 1970's \cite{huffman,resch1974portfolio}. This art, though intimately linked to the mathematics of developable surfaces, is mostly driven by physical experiments with paper \cite{curved_review}. As opposed to smooth developable surfaces \cite{do_carmo} or straight fold origami \cite{origami_book}, the mathematics of curved folding is legging behind the manual craft, and mostly concerns the local behavior of a single folded curved crease \cite{duncan_folded,mathematical_omnibus,curved_review}. Notable crease patterns such as the Huffmann Tower \cite{huffman2,huffmann_reconstructing}, are not yet understood \cite{demaine2018conic} and the known mathematics on folding and bending multiple curved creases is limited to a very few particular cases, coupled with a specific folding movement and guided by fixed rulings \cite{demaine_lens, demaine2018conic}. In essence, we do not know much about which crease patterns fold, and we do not know in which ways they can fold; Unlike straigh origami, there are often infinite ways of folding bending a curved crease pattern by varying the dihedral angles as well as the developable rulings along different creases. \\
Curved folding was first introduced to the geometry processing community by work of \cite{curved_folding_kilian}, where the authors devised an algorithm to reconstruct scanned paper curved folded surfaces by first to estimate their ruling directions. Several works on modeling developable surfaces focus on a given subset of folding deformations, such as planar creases generated by reflection \cite{mitani2012column,Mitani_ref}, surfaces generated by rotational sweeps \cite{mitani2009design} or surface folded with a fixed ruling pattern \cite{pottmann_new}. In \cite{StringActuated:2017} the authors simplify the process of fabrication for a range of well known curved folded models using a network of strings, solving the problem of which surface points to pull in order to actuate a folding movement. To model the folding deformation the authors of \cite{StringActuated:2017} employ the model of \cite{botsch2006primo} with the remeshing algorithm in \cite{narain2012adaptive}. Their deformation is guided by an input folds orientation given as mountain/valley assignments, prescribed constrained folding angles, and a bending objective. We note that prescribing dihdedral angles of a single curve results in an undetermined system, while prescribing the folding angles of multiple folds often results in an overdetermined system: There are infinite number of ways to fold a surface with the same prescribed folding angles \cite{more_on_paper,duncan_folded}, however dihedral angles across multiple folds needs to comply with one another \cite{demaine2018conic}. \\
To the best of the authors' knowledge, the first freeform handle based system for curved folding deformations was introduced by \cite{rabi2018shape}.  The authors of \cite{rabi2018shape} employed multiple DOGs together with a set of stitching and flattability constraints to model curved crease deformations. This work builds and extends upon the work of \cite{rabi2018shape}, by deriving a discrete charecterization for a fold along a crease, allowing us to devise a point based editing system that do folds the creases of a given pattern rather than smoothly ignoring them, all without specifying dihedral angles or folds orientation (see \figref{fig:folded_and_not_folded}). To allow for further control in the folding process, we also derive simple linear constraints to control dihedral angles along folds and a constraint allowing to control mountain/valley assignments, a degree of freedom which is often only available along one curved crease (see \MiR{dihedral figure} and \figref{fig:MV_bias_modeling}).