\section{Related work}
\subsection{Modeling developable surfaces}
A smooth surface is called a \textit{developable surface} if it is locally isometric to the plane, or equivalently has zero Gaussian curvature. Though well understood mathematically \cite{do_carmo,spivak,computational_line}, there is a plethora of work on modeling developable surfaces, which has been proven to be a challenge. The primary difficulty lies in using a discrete model that is able to capture the full set of deformations keeping a surface developable. These include extrinsic deformations as well as intrinsic deformations. The latter stretch the surface while keeping it developable, and is used for geometry exploration tasks where the size and shape of the flattened developable surface is unknown \cite{conical,pottmann_new,rabi2018shape}. A failure of a discrete model to discretize the full range of smooth deformations is often termed \textit{locking} \cite{solomon,locking1} and is the bane of most discrete developable models. Ruling based models \cite{conical,curved_folding_kilian,pottmann_new,stein_dev,solomon} lock the user to a partial set of extrinsic deformations while isometry based methods \cite{grin_shells,shells, goldenthal2007efficient,froh_botsch} do not model intrinsic deformations by design, but are also prone to locking of various bending deformations \cite{locking1,locking2}, and are often coupled with dynamic remeshing \cite{narain2012adaptive,StringActuated:2017,Narain,SchreckEG2017}. \MiR{should I explicitly say that remeshing is complicated and is also not backed up by any theory?} Our work is based on discrete orthogonal geodesic nets \cite{rabi18} (DOGs), shown both theoretically and empirically to avoid extrinsic or intrinsic deformations locking. We further rely on \cite{rabi2018shape} to deform and explore the shape space of DOGs, though we replace the Laplacian flow by an SQP algorithm with a convexified Hessian as further detailed in \secref{sec:implementation}.
\subsection{Curved folding}
The beautiful art of curved folded sculptures is almost a hundred years old, dating back to the 1920's works of Josef Albers in the Bauhaus art school \cite{josef_albers_thesis}, and continuing with the investigations of David Huffman and Ron Resch in the 1970's \cite{huffman,resch1974portfolio}. This art, though intimately linked to the mathematics of developable surfaces, is mostly driven by physical experiments with paper \cite{curved_review}. As opposed to smooth developable surfaces \cite{do_carmo} or straight fold origami \cite{origami_book}, the mathematics of curved folding is legging behind the manual craft, and mostly concerns the local behavior of a single folded curved crease \cite{duncan_folded,mathematical_omnibus,curved_review}. Notable crease patterns such as the Huffmann Tower \cite{huffman2,huffmann_reconstructing}, are not yet understood \cite{demaine2018conic} and the known mathematics on folding and bending multiple curved creases is limited to a very few particular cases, coupled with a specific folding movement and guided by fixed rulings \cite{demaine_lens, demaine2018conic}. In essence, we do not know much about which crease patterns fold, and we do not know in which ways they can fold; Unlike straigh origami, there are often infinite ways of folding bending a curved crease pattern by varying the dihedral angles as well as the developable rulings along different creases. \\
Curved folding was first introduced to the geometry processing community by work of \cite{curved_folding_kilian}, who reconstructed scanned curved folded surfaces. In \cite{StringActuated:2017} the authors simplify the process of fabrication of curved folded models using a string actuation network. Both of these works assume an access to known target surfaces. Other works on modeling developable surfaces focus on a given subset of folds, such as planar creases generated by reflection \cite{mitani2012column,Mitani_ref}, surfaces generated by rotational sweeps \cite{mitani2009design}. The works \cite{pottmann_new} model curved folded surfaces with a rigid ruling pattern. We note that the majority of curved folding deformations change the rulings throughout the folding process. 
% Write that the only system of freeform exploration of curved folding that the authors know of is based on DOGs, that this hits a limit with multiple creases and that we extend upon that work (refer to fig 2 again).