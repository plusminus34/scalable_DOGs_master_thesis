\section{Introduction}
There are myriads of ways to deform a planar sheet without stretching or tearing it. One can bend it, fold it, or combine the two. Folding and bending isometries are different by nature, and historically speaking there is some dichotomy in the study of the two; Smooth isometries are typically studied in differential geometry \cite{do_carmo}, whereas straight folds are often explored in the field of computational origami \cite{origami_book}. Curved folded surfaces \cite{huffman} results from a combination of the two, as folding an inextensible sheet along a curve necessitates global bending around the crease.

Building curved folded sculptures is a manual and time consuming process, often done using an empirical trial and error approach  as little theory is known \cite{curved_review,huffmann_reconstructing}. Contrary to classical origami, bending and folding instructions are hard to write down and the fact that multiple creases need to fold simultaneously further complicates the process \cite{StringActuated:2017}. In practice, artists often pre-crease the paper using a ball burnisher or a CNC plotter before carefully folding and bending it.

Albeit manual, slow, and exhaustive, playing with paper is currently the only option available for an explorer of curved folds. Existing works on modeling these surfaces are either limited to previously discovered sculptures \cite{curved_folding_kilian,StringActuated:2017} or model a small partial set of folded surfaces generated by reflections or rotational sweeps \cite{Mitani_ref,mitani2009design}. Modeling the folding process of novel forms remains a challenge \cite{curved_review}. In this paper we set to develop the basic tools for unconstrained modeling of curved folds, in order to aid the exploration, analysis and study of new curved folded structures.

Our work builds upon Discrete Orthogonal Geodesic Nets (DOGs) \cite{rabi18,rabi2018shape} as a discrete model for smooth developable surfaces parameterized by orthogonal geodesics. These are regular quadrilateral meshes with equal angles around each vertex, and unlike other computational models for developables do not suffer from locking of various deformation modes \cite{locking1,locking2,grin_shells}, are not fixed to predetermined rulings directions \cite{pottmann_new,curved_folding_kilian} or require remeshing while deforming the surface \cite{StringActuated:2017,SchreckEG2017,Narain}. We represent curved folded models as piecewise smooth developable surfaces with tangent discontinuities and equal geodesic curvature along their intersections, as done in \cite{rabi2018shape}. 

% Bifurcation mode... Explains that it doesn't have to fold, and it is more complicated when there are more folds, and hence one needs to give a simple definition and algorithm for folding. Then continue that there are other parameters one would like to control such as angles and rulings, and bla bla.
The challenge is then to find deformations that leads to a folded configuration (ADD:FIGURE).

% There are 2 components for such modeling. The first is a good model, the second is a way to navigate and fold that model, similarly as to how one needs to place is fingers in order to fold a piece of paper. We use the model


% The beautiful art of curved folded sculptures is almost a hundred years old, dating back to the 1920's works of Josef Albers in the Bauhaus art school \cite{josef_albers_thesis}, and continuing with the investigations of David Huffman and Ron Resch in the 1970's \cite{huffman,resch1974portfolio}. 

% Many of these early works, are mathematically not yet fully understood. We know what happens in one fold, but multiple ones still forms a challenge and we only have specific investigation for some cases. Creating them is , using paper and They date to 1920's but we only know what happen along one fold, or analyze specific set of examples, and don't know how typical models even fold. Computer simulations of that can help, but are somehow legging behind what is mostly experimented by hand, in a process that is in fact time consuming, often involving using a pen bla and da da da.
