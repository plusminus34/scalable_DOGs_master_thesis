\section{Controlling the rulings} \label{sec:rulings}

\subsection{Controlling rulings along the curve} 
If the ruling angles of both surfaces along the crease are $\beta_1(t),\beta_2(t)$, measured by their angles with the tangents, then they satisfy the following \cite{more_on_paper,duncan_folded}:

\begin{equation}
\cot\beta_1(t) = \frac{\alpha'(t)-\tau(t)}{k(t)\sin(\alpha(t))},\cot\beta_2(t) = \frac{-\alpha'(t)-\tau(t)}{k(t)\sin(\alpha(t))}
\end{equation}
Which from here we can deduce (\cite{mathematical_omnibus,duncan_folded}):
\begin{equation} \label{cot_eq}
\begin{split}
\cot\beta_1(t) + \cot\beta_2(t) = \frac{-2\tau(t)}{k(t)\sin(\alpha(t))},\\
\cot\beta_1(t) - \cot\beta_2(t) = \frac{2\alpha'(t)}{k(t)\sin(\alpha(t))},
\end{split}	
\end{equation}
The authors of (\cite{mathematical_omnibus,duncan_folded}) identified two special cases, corresponding to when the above equations vanish. The first correspond to a planar curved fold, i.e. $\cot\beta_1(t) + \cot\beta_2(t) = \tau = 0$ which implies $\beta_1+\beta_2 = \pi$, meaning the rulings are continuation of each other on the flattened surface. In that case the two surfaces are just reflections of each other through the unique osculating plane of the curve (which doesn't change as $\tau = 0$). The second case corresponds to constant dihedral angle along the curve, i.e. $\cot\beta_1(t) - \cot\beta_2(t) = \alpha(t)' = 0$ which implies $\beta_1 = \beta_2$. These two cases coincide in the case $\beta_1 = \beta_2 = \frac{\pi}{2}$.


\subsection{Rulings objectives on DOG vertices}
